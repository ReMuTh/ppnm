\documentclass[a4paper]{article}
\usepackage[utf8]{inputenc}
\usepackage{graphicx}
\usepackage{listings}
\usepackage{color}

\title{Approximating the exponential function}
\author{René Munk Thalund}
\date{March 2022}

\begin{document}
\maketitle
\begin{abstract}
We examine a \texttt{C\#} implementation of the exponential function that uses only multiplications and additions.
\end{abstract}

\section{Introduction}
The exponential function $e^x$ is ubiquitous in math and science. It's defined as the unique function that is it's own derivative, and has value 1 for argument 0, that is:

\begin{equation}
    \frac{df}{dx} = f(x) \; \textrm{and} \;  f(0) = 1
\end{equation}
The definition extends to complex arguments. Notably with an entirely imaginary argument $e^{i\phi} Z$ rotates the complex number Z by $\phi$ radians in the complex plane.


\section{Approximation}
We're considering the following \texttt{C\#}-function.

\begin{lstlisting}{language=[Sharp]C}
static double ex(double x){
    if(x<0)return 1/ex(-x);
    if(x>1.0/8)return Pow(ex(x/2),2);
    return 1+x*(1+x/2*(1+x/3*(1+x/4*(1+x/5*(1+x/
    6*(1+x/7*(1+x/8*(1+x/9*(1+x/10)))))))));
}
\label{code}
\end{lstlisting}
For small arguments $e^x$ is approximated using the geometric series up to the ninth term:

\begin{equation}
    \label{eq:regime1}
    e^x \approx \sum_{n = 0}^9 \frac{x^n}{n!} \; ,\;\;\; x \in [0,0.125]
\end{equation}
In the source code the sum us convoluted so the minimum number of multiplications and additions is used. For larger arguments $e^x$ is evaluated recursively, using the algebraic identity:

\begin{equation}
    \label{eq:regime2}
    e^x = (e^{x/2})^2, \;\; \textrm{used for}\;\;  x > 0.125
\end{equation}
As seen the argument is halfed in each recursion step until the regime defined by eq. \ref{eq:regime1} is reached. For negative arguments $e^x$ is evaluated as the reciprocal of the correponding positive argument:

\begin{equation}
    \label{eq:regime3}
    e^{-x} = \frac{1}{e^x} \;\; \textrm{used for}\;\;  x < 0
\end{equation}


\begin{figure}
    \input{figures/ex_plot.tex}
    \caption{The approximation is very precise}
    \label{fig:result}
\end{figure}


\section{Results}
For moderate values of x the approximation is very precise. The deviation of $ex(x)$ from $exp(x)$ is so small we have no chance of observing the difference in a direct plot of the values as in fig. \ref{fig:result}.



\begin{figure}
   \input{figures/exerror_plot.tex}
    \caption{The relative error rises in steps due to the recursive definition of the approximation}
   \label{fig:result2}
\end{figure}

Instead we consider the relative deviation from the system math value of $exp(x)$, see fig. \ref{fig:result2}. We notice that the deviation is larger for larger absolute values of x. The deviation rises in steps, which is natural since the approximation for each octave interval $]2^{n},2^{n+1}]$ is inherited from the previous down to the 'mould' of the approximation at $]2^{-4},2^{-3}]$. Since
\begin{equation}
    \frac{ex(2x)}{exp(2x)} = \left( \frac{ex(x)}{exp(x)} \right)^2
\end{equation}
we can estimate the deviation at $e^{709} \approx 1 \times 10^{308}$, the largest double allowed in \texttt{C\#}. The approximation for this value is six 'generations' up from the the $]8,16]$-interval which we read from fig. \ref{fig:result2} to have a relative deviation at about $1 + 1.5 \times 10^{-14}$. Hence, using
$(1 + \epsilon)^2 \approx (1 + 2 \epsilon)$ the relative deviation at the maximum calculable value will be about $1 + 9.6 \times 10^{-13}$. Not that bad.




\section{Conclusion}
We have successfully shown that $e^x$ can be estimated using only multiplications and additions. 


\end{document}